\documentclass{article}
\usepackage[utf8]{inputenc}

\title{Movimiento de proyectiles}
\author{Carlos David Ureña Pérez }
\date{30 de agosto del 2019}

\usepackage{natbib}
\usepackage{graphicx}

\begin{document}

\maketitle

\section{Introduccion}

El movimiento de proyectil es una forma de movimiento que experimenta un objeto o partícula (un proyectil ) que se arroja cerca de la superficie de la Tierra y se mueve a lo largo de una trayectoria curva solo bajo la acción de la gravedad (en particular, se supone que los efectos de la resistencia del aire son insignificantes ) Galileo demostró que este camino curvo es una parábola.

\begin{wrapfigure}
\begin{center}
\includegraphics[height=5cm]{a.jpeg}
\end{center}
\end{wrapfigure}
\caption{Ejemplo de tiro parabólico}
\centering

\section{Tiempo de vuelo}
El tiempo de vuelo total que el proyectil permanece en movimiento.
Para encontrar esto, tenemos en cuenta que y=0 cuando el cuerpo llega al suelo.
\begin{equation}
t=\frac{2vo\sin(\alpha)}{g}
\end{equation}\
En el ejemplo que veremos a continuación se vio la trayectoria de un proyectil desde una altura 0, con un ángulo del disparo de 45 y una velocidad inicial de 12m/s:
\begin{wrapfigure}
\begin{center}
\includegraphics[height=5cm]{tiempo de vuelo.jpeg}
\end{center}
\end{wrapfigure}
Como se puede observar en el ejemplo el tiempo total del recorrido del proyectil es de 1.73 segundos.
\section{Alcance máximo o rango del proyectil}
El alcance máximo es la distancia horizontal que recorre el proyectil. Obtenemos la ecuación al sustituir la expresión del tiempo de movimiento en la ecuacion de la coordenada x:
\begin{equation}
    x=voxt=vo \cos (\alpha) \frac{2vo\sin (\alpha)}{g}=\frac{2vo^2\sin (\alpha)\cos(\alpha)}{g}
\end{equation}
Y utilizando la relación trigonométrica \sin 2\alpha=2\sin \alpha \cos \alpha , resulta:
\\
\begin{equation}
    x=\frac{vo^2}{g}\sin 2 \alpha
\end{equation}
\section{Altura máxima}
La altura máxima de un proyectil lanzado se alcanza cuando vy=0. De aquí podemos obtener el valor de t.
\begin{equation}
    t=\frac{voy}{g}=\frac{vo\sin (\alpha)}{g}
\end{equation}
Sustituimos este valor en la ecuación de la coordenada y:
\begin{equation}
    ymax=voyt-\frac{1}{2}gt^2=\frac{vo^2\sin ^2 \alpha}{g}-\frac{vo^2 \sin ^2 \alpha}{2g}
\end{equation}
\begin{equation}
    ymax=\frac{vo^2\sin ^2 \alpha}{2g}
\end{equation}
La altura máxima que se puede obtener cuando se dispara un proyectil solo se puede obtener cuando en este caso el "cañón" se encuentra a 90 grados, como se puede observar en el siguiente ejemplo:
\begin{wrapfigure}
\includegraphics[height=5cm]{aa.jpeg}
\end{wrapfigure}
En esta primera imagen vemos que el cañón tiene una inclinación de 90 grados y una velocidad inicial de 20m/s y como resultado logra alcanzar una altura de 20.39m
\begin{wrapfigure}
\includegraphics[height=5cm]{aaa.jpeg}
\end{wrapfigure}
Caso contrario tenemos este disparo del cañón con una inclinación de 45 grados y con la misma velocidad inicial de 20m/s. Como resultado en el punto más alto de la parábola generada por el recorrido su altura es de solo 10.19m

\section{Referencias}
\begin{enumerate}
    \item Proyectile motion - Wikipedia: https://en.wikipedia.org/wiki/Projectile-motion
    \item Tiro Parabolico - Descartes 2D: http://recursostic.educacion.es/descartes/web/materiales-didacticos/comp-movimientos/parabolico.htm
    
\end{enumerate}
\end{document}
